\newcommand{\demopagedoubleinput}[4]{
	% #1 what file to render
	% #2 what file to write
	% #3 render minipage width
	\begin{frame}
	\begin{multicols}{2}
		\begin{overlayarea}{#3}{.8\textheight}
			\input{#1}
		\end{overlayarea}
	\columnbreak
		\resizebox{\linewidth}{!}{
			\begin{overlayarea}{\linewidth}{.8\textheight}
				\lstinputlisting[#4]{#2}
			\end{overlayarea}
		}
	\end{multicols}
	\end{frame}
}
\newcommand{\demopageinput}[3]{
	% #1 what file to render
	% #2 render minipage width
	\demopagedoubleinput{#1}{#1}{#2}{#3}
}
\newcommand{\demoinput}[1]{
	% #1 what file to render
	\input{#1} \pause
	\lstinputlisting{#1}
}
\newenvironment{demoareas}{
	\begin{multicols}{2}
	\centering
	\begin{overlayarea}{\resultwidth\textwidth}{.8\textheight}
}{
	\end{overlayarea}
	\end{multicols}
}
\newcommand{\demoareasmiddle}{
	\end{overlayarea}
	\columnbreak
	\begin{overlayarea}{.5\textwidth}{.8\textheight}
}
%shamelessly stolen from https://tex.stackexchange.com/questions/298133/using-lstinline-inside-a-new-command-or-environment#298151
\NewDocumentCommand\demoline{v}{%
	\scantokens{#1\noexpand} & \lstinline{#1} \\ %
}
\NewDocumentCommand\demo{v}{%
	\scantokens{#1\noexpand} \\ \begin{center}
		\lstinline{#1}
	\end{center} %
}

\newcommand{\drawgrid}[4]{
	%draw a background grid in a tikzpicture
	%environment in the rectangle
	%from (#1, #2) to (#3, #4)
	\draw[style={help lines, color=blue!50}] (#1, #2) grid (#3, #4);
	\draw (0, #2) -- (0, #4);
	\draw (#1, 0) -- (#3, 0);
}


%only helper commands so the tikz becomes readable
\newcommand{\ellipseX}{cos(\x r)*1.5}
\newcommand{\ellipseY}{0.5+sin(\x r)*0.5}
\newcommand{\inverseX}[2]{((#1)/((#1)^2 + (#2)^2))}
\newcommand{\inverseY}[2]{((#2)/((#1)^2 + (#2)^2))}
\newcommand{\inverseellipseX}{\inverseX{\ellipseX}{\ellipseY}}
\newcommand{\inverseellipseY}{\inverseY{\ellipseX}{\ellipseY}}
