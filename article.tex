\documentclass{article}
\usepackage[utf8]{inputenc}
\usepackage{xcolor}             % \textcolor ab section Text-1
\usepackage{amsmath}            % \text
\usepackage{listings}           % \lstinline Anleitung section Mathe-2
\usepackage[german]{babel}      % „ und “: \glqq und \grqq
\usepackage{tikz}               % für Zeichnung
\usepackage{hyperref}           % \url
\usepackage{tipa}               % text IPA, used by ODextraletters
%\def\ODdetails#1#2#3{
	% #1: ascii-written dnk
	% #2: includegraphics height
	% #3: command args
	\IfFileExists{OD_letters/OD_#1.pdf}{
		\raisebox{-.2ex}{\includegraphics[height=#2]{OD_letters/OD_#1}}
	}{
		\typeout{notfound OD_letters/OD_#1.pdf}
		\input|"./getletter.py '#1' #3"
	}
}
\def\OD #1 { \ODdetails{#1}{2ex}{}}
\def\ODlarge #1 { \ODdetails{#1}{3ex}{} \hspace{-3ex}}

\def\lexiconentryEN#1#2#3{
	{\bf#1}
	\phantom{.}
	#2
	#3
}
\def\lexiconentryDK#1#2#3#4#5{
	%#1: romanization
	%#2: glyph filename
	%#3: grammatical
	%#4: translation
	%#5: extra
	{\bf#1}
	\hspace{-2ex} \OD{#2} \hspace{-.5ex}
	\ifx#3\empty %nothing
	\else
		% \phantom{.}
		{\it#3}
	\fi
	\ifx#4\empty %
	\else
		% \phantom{.}
		#4
	\fi
	\ifx#5\empty %
	\else
		{(#5)}
	\fi
}

\def\exampleentryDK#1#2#3{
	%#1: romanization
	%#2: dwarven letter commands like \OD[]{...} \OD[]{...}
	%#3: translation
	#2 \\
	{\bf#1} \\
	\ifx#3\empty %
	\else
		% \phantom{.}
		#3
	\fi \\
}

%../../../WORLDS/Alaia/lang/latex_setup/extraletters.tex
%\definecolor{codegreen}{rgb}{0,0.6,0}
\definecolor{codegray}{rgb}{0.5,0.5,0.5}
\definecolor{codepurple}{rgb}{0.58,0,0.82}
\definecolor{backcolour}{rgb}{0.95,0.95,0.92}

\lstdefinestyle{myLaTeXstyle}{
	backgroundcolor=\color{backcolour},
	commentstyle=\color{codegreen},
	keywordstyle=\color{magenta},
	% numberstyle=\tiny\color{codegray},
	stringstyle=\color{codepurple},
	basicstyle=\footnotesize\ttfamily,
	breakatwhitespace=false,
	breaklines=true,
	captionpos=b,
	keepspaces=true,
	% numbers=left,
	% numbersep=5pt,
	% prebreak=\raisebox{0ex}[0ex][0ex]{\tiny\ensuremath{\hookleftarrow}},
	showspaces=false,
	showstringspaces=false,
	showtabs=false,
	tabsize=2,
	language=[LaTeX]TeX
}

\lstdefinestyle{myLaTeXstyleTiny}{
	backgroundcolor=\color{backcolour},
	commentstyle=\color{codegreen},
	keywordstyle=\color{magenta},
	stringstyle=\color{codepurple},
	basicstyle=\tiny\ttfamily,
	breakatwhitespace=false,
	breaklines=true,
	captionpos=b,
	keepspaces=true,
	showspaces=false,
	showstringspaces=false,
	showtabs=false,
	tabsize=2,
	language=[LaTeX]TeX
}

\lstset{style=myLaTeXstyle}
					% listings configuration

\title{\LaTeX{}-Basics}
\author{Antonia}
\date{Oktober 2024}

\begin{document}

\maketitle

\section{Basics}
\subsection{Text}
Hier ein bisschen
\textbf{Text},
teils \emph{betont},
teils \underline
{unterstrichen}).

Zeilenumbruch entweder
durch eine leere Zeile
oder durch `\\'.

\textcolor{red}{Textfarbe}
ist auch kein
{\color{blue} Problem}.

\subsection{Mathe}
$ v = (x, y) $ \\
$ |\vec{v}| = \sqrt{x^2 + y^2} $ \\
$ \sqrt[3]{8} = 2 $ \\
$ x_1 = 10^{-2}\text{m} $ \\
$ \sum_{n=1}^\infty \frac{1}{2^n} = 1 $ \\
$$ \sum_{n=1}^\infty \frac{1}{2^n} = 1 $$


Baut das hier nach, testet die Effekte von einfachen Dollar und doppelten Dollar. Schlagt bekannte Symbole nach (z.B. in der Wikipedia: Liste Mathematischer Symbole), lasst unbekannte notfalls weg ($\zeta$ ist \textbackslash\lstinline+zeta+, $\prod$ \textbackslash\lstinline+prod+) Aufrechten Text kann man mit \textbackslash\lstinline+rm+ erzwingen.
$$ \zeta(s)
	= \sum_{n=1}^\infty \frac{1}{n^s}
	= \prod_{p \in \text{Prim}} \frac{1}{1-p^{-s}}
$$
Stellt Fragen.

Die Fortpflanzung der Messunsicherheit $\sigma_{x_i}$ der Einflusswerte $x_i$ auf $\sigma_y$ mit $y = f(x_0, ..., x_n)$ berechnet sich nach {\sc Gauss} als $\sigma_y = $
$$ |\nabla f(x_{...}) \cdot (\sigma_{x_{...}})| =
\left|
	\begin{pmatrix} \frac{\partial f(x_{...})}{\partial x_0} \\ \vdots \\ \frac{\partial f(x_{...})}{\partial x_n} \end{pmatrix} \cdot
	\begin{pmatrix} \sigma_{x_0} \\ \vdots \\ \sigma_{x_n} \end{pmatrix}
\right| = \sqrt{\sum_{i=0}^n {\left(\frac{\partial f(x_{...})}{\partial x_i} \cdot \sigma_{x_i}\right)}^2} $$

\section{Verschiedenes}
\subsection{Eigene Befehle}
Folgender Code kommt immer wieder vor und soll ein bisschen abstrahiert werden:
\lstinline+$$ \frac{ \partial f(x_{...}) }{ \partial x_0 } $$+

$$ \frac{ \partial f(x_{...}) }{ \partial x_0 } $$

\newcommand{\partialdiff}[2]{ % {\name}[arg-count]
	\frac{\partial #1}{\partial #2} % definition
}
$$ \partialdiff{ f(x_{...}) }{ x_0 } $$

\hrule\vspace{1em}
\glqq Alle Strukturen, die mehr als einmal vorkommen und alle Variablen werden separat definiert.\grqq{} -- meine Tante

\subsection{Umgebungen}
Die folgende Tabelle wird an den Anfang der Seite wandern:
\begin{table}
\begin{tabular}{r|l}
	Umgebung & Sinn \\
	\hline
	{\tt tabular} & Form \\
	{\tt table} & Kontext \\
	{\tt figure} & Abb. \\
	{\tt matrix} & Matrix \\
	{\tt itemize} & Listen \\
\end{tabular}
	\label{tab:environments}
	\caption{Einige Umgebungen}
\end{table}

Formatierte mathematische Formeln:
\begin{align*}
	\vec{v} &= (x, y) \\
	a   &= \sqrt[3]{8} = 2 \\
	x_1 &= 10^{-2}\text{m} &= 1\text{cm} \\
\end{align*}

\begin{figure}
$$ \zeta(s)
	= \sum_{n=1}^\infty \frac{1}{n^s}
	= \prod_{p \in \text{Prim}} \frac{1}{1-p^{-s}}
$$
\caption{Eine Formel}
\label{fig:zeta}
\end{figure}

\newcommand{\drawgrid}[4]{
	%draw a background grid in a tikzpicture
	%environment in the rectangle
	%from (#1, #2) to (#3, #4)
	\draw[style={help lines, color=blue!50}] (#1, #2) grid (#3, #4);
	\draw (0, #2) -- (0, #4);
	\draw (#1, 0) -- (#3, 0);
}
\newcommand{\ellipseX}{cos(\x r)*1.5}
\newcommand{\ellipseY}{0.5+sin(\x r)*0.5}
\newcommand{\inverseX}[2]{((#1)/((#1)^2 + (#2)^2))}
\newcommand{\inverseY}[2]{((#2)/((#1)^2 + (#2)^2))}
\newcommand{\inverseellipseX}{\inverseX{\ellipseX}{\ellipseY}}
\newcommand{\inverseellipseY}{\inverseY{\ellipseX}{\ellipseY}}

\subsection{Zeichnungen}
\begin{figure}
\begin{tikzpicture}[scale=1.6]
	\drawgrid{-2.5}{-2.5}{2.5}{2.5}
	\draw[thick] (0, 0) circle [radius=1];
	\draw[thick, red] plot[domain=0:360, samples=40]
		({cos(\x)*1.5},{0.5+sin(\x)*0.5});
	\draw[thick, blue] plot[domain=-1.301:4.444,
		samples=150] ({\inverseellipseX},
		{\inverseellipseY});
\end{tikzpicture}
	\caption{ $E(x, y, z)$ }
	\label{fig:ellipse}
\end{figure}

Verweise auf {\tt label} mit {\tt ref}: \\
Siehe Abbildung \ref{fig:ellipse},
	Tabelle \ref{tab:environments}


%\exampleentryDK{d\'aw.\-\sh{}y \gh{}\'y\sh{}.\-fa.\-g\`en}
	%{\ODlarge{da2w.s1y} \ODlarge{g1y2s1.fa.ge1n}}
	%{They blossom into a healthy young dwarf}
%
%\exampleentryDK{d\'ed.\-t\zh{}ez b\'or.\-mbez.\-\`ar nd\=em}
	%{\ODlarge{de2d.tz1ez} \ODlarge{bo2r.mbez.a1r} \ODlarge{nde3m}}
	%{I have written 10 books}
%
%\exampleentryDK{d\=el w\'af.\-es 7 g\'e\zh{}.\-t\sh{}ew.\-t\`o}
	%{\ODlarge{de3l} \ODlarge{wa2f.es} \ODlarge{7} \ODlarge{ge2z1.ts1ew.to1} }
	%{Seven rivers flow through the Ge\zh{}.t\sh{}ew}
%
%\exampleentryDK{d\=ew \Sh{}\'ef.\-\nh\gh{}ow.\-\`es t\'e\gh{}.\-or \zh{}ram.\-x\'ar.\-gys \gh{}\=ep}
	%{\ODlarge{de3w} \ODlarge{S1e2f.n1g1ow.e1s} \ODlarge{te2g1.or} \ODlarge{z1ram.xa2r.gys} \ODlarge{g1e3p}}
	%{The \Sh{}ef\nh\gh{}ows have developed their mapmaking for 100 years}
%
%\exampleentryDK{nzy.\-dl\'ym.\-xal \zh{}\'yw.\-ar ra\zh{}.\-nt\'aw}
	%{\ODlarge{nzy.dly2m.xal} \ODlarge{z1y2w.ar} \ODlarge{raz1.nta2w}}
	%{They will teach you everything you need}

\begin{itemize}
	\item[Slides] \url{users.ifsr.de/~antonia.obersteiner/LaTeX.pdf}
	\item[TUD] Templates der Uni und verschiedener Lehrstühle
	\item[TUD-OL] \url{https://tex.zih.tu-dresden.de} (Overleaf der Uni)
	\item[SE] \url{https://tex.stackexchange.com} (Q\&A-Portal)
	\item[OL] \url{https://overleaf.com} (Tipps und Editor)
	\item[Tikz] \url{https://tikz.dev/}
		(Zeichnungen $\to$ \hyperlink{tikz-final}{Slide})
	\item[Mathe] \url{https://de.wikipedia.org/wiki/Liste_mathematischer_Symbole}
	\item[Symbole] \url{https://latex-programming.fandom.com/wiki/List_of_LaTeX_symbols}
\end{itemize}

\end{document}

