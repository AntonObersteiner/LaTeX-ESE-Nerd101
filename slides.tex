\documentclass{beamer}
\usepackage{geometry}
\usepackage[german]{babel}
\usepackage{listings}      %listings of code
\usepackage{multicol}
\usepackage{tikz} %draw

\usetheme{Dresden}

\newcommand{\demopagedoubleinput}[4]{
	% #1 what file to render
	% #2 what file to write
	% #3 render minipage width
	% #4 lst-style changes
	\begin{frame}
	\begin{multicols}{2}
		\begin{overlayarea}{#3}{.8\textheight}
			\input{#1}
		\end{overlayarea}
	\columnbreak
		\resizebox{\linewidth}{!}{
			\begin{overlayarea}{\linewidth}{.8\textheight}
				\lstinputlisting[#4]{#2}
			\end{overlayarea}
		}
	\end{multicols}
	\end{frame}
}
\newcommand{\demopagedoubleinputmeh}[5]{
	% #1 what file to render
	% #2 what file to write
	% #3 render minipage width
	% #4 lst-style changes
	% #5 listing width factor
	\begin{frame}
	\begin{multicols}{2}
		\begin{overlayarea}{#3}{.8\textheight}
			\input{#1}
		\end{overlayarea}
	\columnbreak
		\resizebox{\linewidth}{!}{
			\begin{overlayarea}{#5\linewidth}{.8\textheight}
				\lstinputlisting[#4]{#2}
			\end{overlayarea}
		}
	\end{multicols}
	\end{frame}
}
\newcommand{\demopageinput}[3]{
	% #1 what file to render
	% #2 render minipage width
	\demopagedoubleinput{#1}{#1}{#2}{#3}
}
\newcommand{\demoinput}[1]{
	% #1 what file to render
	\input{#1} \pause
	\lstinputlisting{#1}
}
\newenvironment{demoareas}{
	\begin{multicols}{2}
	\centering
	\begin{overlayarea}{\resultwidth\textwidth}{.8\textheight}
}{
	\end{overlayarea}
	\end{multicols}
}
\newcommand{\demoareasmiddle}{
	\end{overlayarea}
	\columnbreak
	\begin{overlayarea}{.5\textwidth}{.8\textheight}
}
%shamelessly stolen from https://tex.stackexchange.com/questions/298133/using-lstinline-inside-a-new-command-or-environment#298151
\NewDocumentCommand\demoline{v}{%
	\scantokens{#1\noexpand} & \lstinline{#1} \\ %
}
\NewDocumentCommand\demo{v}{%
	\scantokens{#1\noexpand} \\ \begin{center}
		\lstinline{#1}
	\end{center} %
}

\newcommand{\drawgrid}[4]{
	%draw a background grid in a tikzpicture
	%environment in the rectangle
	%from (#1, #2) to (#3, #4)
	\draw[style={help lines, color=blue!50}] (#1, #2) grid (#3, #4);
	\draw (0, #2) -- (0, #4);
	\draw (#1, 0) -- (#3, 0);
}


%only helper commands so the tikz becomes readable
\newcommand{\ellipseX}{cos(\x r)*1.5}
\newcommand{\ellipseY}{0.5+sin(\x r)*0.5}
\newcommand{\inverseX}[2]{((#1)/((#1)^2 + (#2)^2))}
\newcommand{\inverseY}[2]{((#2)/((#1)^2 + (#2)^2))}
\newcommand{\inverseellipseX}{\inverseX{\ellipseX}{\ellipseY}}
\newcommand{\inverseellipseY}{\inverseY{\ellipseX}{\ellipseY}}

\definecolor{codegreen}{rgb}{0,0.6,0}
\definecolor{codegray}{rgb}{0.5,0.5,0.5}
\definecolor{codepurple}{rgb}{0.58,0,0.82}
\definecolor{backcolour}{rgb}{0.95,0.95,0.92}

\lstdefinestyle{myLaTeXstyle}{
	backgroundcolor=\color{backcolour},
	commentstyle=\color{codegreen},
	keywordstyle=\color{magenta},
	% numberstyle=\tiny\color{codegray},
	stringstyle=\color{codepurple},
	basicstyle=\footnotesize\ttfamily,
	breakatwhitespace=false,
	breaklines=true,
	captionpos=b,
	keepspaces=true,
	% numbers=left,
	% numbersep=5pt,
	% prebreak=\raisebox{0ex}[0ex][0ex]{\tiny\ensuremath{\hookleftarrow}},
	showspaces=false,
	showstringspaces=false,
	showtabs=false,
	tabsize=2,
	language=[LaTeX]TeX
}

\lstdefinestyle{myLaTeXstyleTiny}{
	backgroundcolor=\color{backcolour},
	commentstyle=\color{codegreen},
	keywordstyle=\color{magenta},
	stringstyle=\color{codepurple},
	basicstyle=\tiny\ttfamily,
	breakatwhitespace=false,
	breaklines=true,
	captionpos=b,
	keepspaces=true,
	showspaces=false,
	showstringspaces=false,
	showtabs=false,
	tabsize=2,
	language=[LaTeX]TeX
}

\lstset{style=myLaTeXstyle}


%thumbnail example in pagenumers.tex
%dummy definitions for section page numbers
\def \pagenumberadvanced {30}
\def \pagenumberdefine {30}
\def \pagenumberdwarven {30}

%file that will contain sectionnumber definitions the import here is
%	the output from the file of the last compilation
\def \pagenumberdefine {14}
\def \pagenumberdefine {15}
\def \pagenumberdefine {16}
\def \pagenumberdefine {17}
\def \pagenumberadvanced {33}
\def \pagenumberadvanced {34}
\def \pagenumberdwarven {35}

%written in the format \newcommand\sectionpagenumber…{…}, where the section labels are defined
\newwrite\sectionpagenumbers
\immediate\openout\sectionpagenumbers=sectionpagenumbers.tex
\newcounter{beamerpagenumber}

%the following two commands
%create a label for the current page and
%write out the current page number to the file sectionpagenumbers.tex
\def\addsectionlabelandpagenumber#1#2{
	\label<\value{beamerpauses}>{#1}
	\immediate\write\sectionpagenumbers{\unexpanded{\def#2}{\thepage}}
}
%difference to the previous: this version adds one to the label's overlay indicator
%this is sometimes useful because for reasons that are beyond me the link sometimes
%shows the right version, but still leads to the overlay before the wanted
\def\addsectionlabelandpagenumbernext#1#2{
	\setcounter{beamerpagenumber}{\value{beamerpauses}}
	\stepcounter{beamerpagenumber}
	\label<\thebeamerpagenumber>{#1}
	\immediate\write\sectionpagenumbers{\unexpanded{\def#2}{\thepage}}
}

\def\minislide#1#2{
	\IfFileExists{./\jobname.old.pdf}{
		\hyperlink{#2}{\includegraphics[page=#1,scale=.25]{\jobname.old.pdf}}
	}{	%else
		\GenericError{[mycode] }{Missing .old copy}%
			{[mycode] File ./\jobname.old.pdf not found}%
			{copy ./\jobname.pdf to ./\jobname.old.pdf and run again \MessageBreak
				(just press <return> to get to ./\jobname.pdf initially) \MessageBreak
				afterwards: always copy ./\jobname.pdf to its .old version to use minislides}%%%
	}
}

%old dwarven example
\def\ODdetails#1#2#3{
	% #1: ascii-written dnk
	% #2: includegraphics height
	% #3: command args
	\IfFileExists{OD_letters/OD_#1.pdf}{
		\raisebox{-.2ex}{\includegraphics[height=#2]{OD_letters/OD_#1}}
	}{
		\typeout{notfound OD_letters/OD_#1.pdf}
		\input|"./getletter.py '#1' #3"
	}
}
\def\OD #1 { \ODdetails{#1}{2ex}{}}
\def\ODlarge #1 { \ODdetails{#1}{3ex}{} \hspace{-3ex}}

\def\lexiconentryEN#1#2#3{
	{\bf#1}
	\phantom{.}
	#2
	#3
}
\def\lexiconentryDK#1#2#3#4#5{
	%#1: romanization
	%#2: glyph filename
	%#3: grammatical
	%#4: translation
	%#5: extra
	{\bf#1}
	\hspace{-2ex} \OD{#2} \hspace{-.5ex}
	\ifx#3\empty %nothing
	\else
		% \phantom{.}
		{\it#3}
	\fi
	\ifx#4\empty %
	\else
		% \phantom{.}
		#4
	\fi
	\ifx#5\empty %
	\else
		{(#5)}
	\fi
}

\def\exampleentryDK#1#2#3{
	%#1: romanization
	%#2: dwarven letter commands like \OD[]{...} \OD[]{...}
	%#3: translation
	#2 \\
	{\bf#1} \\
	\ifx#3\empty %
	\else
		% \phantom{.}
		#3
	\fi \\
}

../../../WORLDS/Alaia/lang/latex_setup/extraletters.tex
\usepackage{tipa}	% ipa characters for OD

% \setbeamercovered{transparent}
\renewcommand{\arraystretch}{1.2} %table line height
\newcommand{\resultwidth}{.4}
\newcommand{\resultwidthintro}{.5}

\title{\LaTeX{}-Basics}
\author{Antonia Obersteiner}
\subtitle{beim ESE-Nerd101}

\begin{document}

\demopagedoubleinputmeh{intro.tex}{intro.lst.tex}{\resultwidthintro\textwidth}{}{1.1}
\demopagedoubleinput{tableofcontents.tex}{tableofcontents.lst.tex}{\resultwidthintro\textwidth}{}

\section{Basics}
\subsection{Text}
\begin{frame}
\begin{demoareas}
	\vspace{2em}
	\only<1>{Hier ein bisschen
\textbf{Text},
teils \emph{betont},
teils \underline
{unterstrichen}).
}
	\only<2>{Hier ein bisschen
\textbf{Text},
teils \emph{betont},
teils \underline
{unterstrichen}).

Zeilenumbruch entweder
durch eine leere Zeile
oder durch `\\'.
}
	\only<3>{Hier ein bisschen
\textbf{Text},
teils \emph{betont},
teils \underline
{unterstrichen}).

Zeilenumbruch entweder
durch eine leere Zeile
oder durch `\\'.

\textcolor{red}{Textfarbe}
ist auch kein
{\color{blue} Problem}.
}
\demoareasmiddle % separate the ``columns''
	\only<1>{\lstinputlisting{text1.tex}}
	\only<2>{\lstinputlisting{text12.tex}}
	\only<3>{\lstinputlisting{text123.tex}}
\end{demoareas}
\end{frame}

\subsection{Mathe}
\begin{frame}[fragile]
\centering
\begin{tabular}{rl}
	\demoline{$ v = (x, y) $} \pause
	\demoline{$ |\vec{v}| = \sqrt{x^2 + y^2} $} \pause
	\demoline{$ \sqrt[3]{8} = 2 $} \pause
	\demoline{$ x_1 = 10^{-2}\text{m} $} \pause
	\demoline{$ \sum_{n=1}^\infty \frac{1}{2^n} = 1 $} \pause
\end{tabular}
\demo{$$ \sum_{n=1}^\infty \frac{1}{2^n} = 1 $$}
\end{frame}

\begin{frame}
\demoinput{mathsigma.tex}
\end{frame}


\section{Verschiedenes}
\subsection{Struktur}
\subsubsection{Definitionen}
\begin{frame}
	\lstinputlisting{partial.tex}
	\pause \vspace{1em}
	\newcommand{\partialdiff}[2]{ % {\name}[arg-count]
	\frac{\partial #1}{\partial #2} % definition
}
$$ \partialdiff{ f(x_{...}) }{ x_0 } $$

	\pause
	\lstinputlisting{definepartial.tex}
	\pause \hrule\vspace{1em}
	\glqq Alle Strukturen, die mehr als einmal vorkommen und alle Variablen werden separat definiert\grqq{} -- meine Tante

	\addsectionlabelandpagenumber{sec:define}{\pagenumberdefine}
\end{frame}

\subsubsection{Input von Dateien}
\begin{frame}
	\begin{multicols}{2}
		\begin{minipage}{.5\textwidth}
			\lstinputlisting{input1.tex}
		\end{minipage}
		\pause
		\lstinline{preamble.tex:}
		\lstinputlisting{preamble.tex}
	\end{multicols}
\end{frame}

\subsubsection{Umgebungen}
\begin{frame}[fragile]
\begin{multicols}{2}
\centering
\begin{overlayarea}{\resultwidth\textwidth}{.8\textheight}
	\only<+>{
\begin{tabular}{r|l}
	Umgebung & Sinn \\
	\hline
	{\tt tabular} & Form \\
	{\tt table} & Kontext \\
	{\tt figure} & Abb. \\
	{\tt matrix} & Matrix \\
	{\tt itemize} & Listen \\
\end{tabular}
}
	\only<+>{
\begin{tabular}{r|l}
	Umgebung & Sinn \\
	\hline
	{\tt tabular} & Form \\
	{\tt table} & Kontext \\
	{\tt figure} & Abb. \\
	{\tt matrix} & Matrix \\
	{\tt itemize} & Listen \\
\end{tabular}

\begin{align*}
	\vec{v} &= (x, y) \\
	a   &= \sqrt[3]{8} = 2 \\
	x_1 &= 10^{-2}\text{m} &= 1\text{cm} \\
\end{align*}
}
	\only<+>{
\begin{tabular}{r|l}
	Umgebung & Sinn \\
	\hline
	{\tt tabular} & Form \\
	{\tt table} & Kontext \\
	{\tt figure} & Abb. \\
	{\tt matrix} & Matrix \\
	{\tt itemize} & Listen \\
\end{tabular}
}
	\only<+>{\begin{table}
\begin{tabular}{r|l}
	Umgebung & Sinn \\
	\hline
	{\tt tabular} & Form \\
	{\tt table} & Kontext \\
	{\tt figure} & Abb. \\
	{\tt matrix} & Matrix \\
	{\tt itemize} & Listen \\
\end{tabular}
	\label<4>{tab:environments}
	\caption{Einige Umgebungen}
\end{table}
}
\end{overlayarea}
\begin{overlayarea}{.5\textwidth}{.8\textheight}
	\vspace{-1.5em}
	\only<1>{\lstinputlisting{table1.tex}}
	\only<2>{\lstinputlisting{table12.tex}}
	\only<3>{\lstinputlisting{table1.tex}}
	\only<4>{\lstinputlisting{table123.tex}}
\end{overlayarea}
\end{multicols}
\end{frame}


\subsection{Übung 2}

\begin{frame}
	In gewöhnlichen Dokumenten sortieren sich Text und Abbildungen je nach Konfiguration auseinander. Testet das und findet dann heraus, wie man es unterbinden kann.
	\begin{figure}
	$$ \zeta(s)
		= \sum_{n=1}^\infty \frac{1}{n^s}
		= \prod_{p \in \text{Prim}} \frac{1}{1-p^{-s}}
	$$
	\caption{Eine Formel}
	\label{eq:first}
	\end{figure}
	In Präsentationen bleibt Abbildung \ref{eq:first} einfach zwischen dem Text stehen.
\end{frame}


\subsection{Zeichnen}
\begin{frame}[fragile]
\begin{tabular}{rl}
\begin{overlayarea}{.3\textwidth}{.8\textheight}
	\only<1,2>{
\begin{tikzpicture}[scale=.6]
	\drawgrid{-2.5}{-2.5}{2.5}{2.5}
	\draw[thick] (0, 0) circle [radius=1];
\end{tikzpicture}
}
	\only<3>{
\begin{tikzpicture}[scale=.6]
	\drawgrid{-2.5}{-2.5}{2.5}{2.5}
	\draw[thick] (0, 0) circle [radius=1];
	\draw[thick, red] plot[domain=0:360, samples=40] ({cos(\x)*1.5},{0.5+sin(\x)*0.5});
\end{tikzpicture}
}
	\only<4>{
\begin{tikzpicture}[scale=.6]
	\drawgrid{-2.5}{-2.5}{2.5}{2.5}
	\draw[thick] (0, 0) circle [radius=1];
	\draw[thick, red] plot[domain=0:360, samples=40] ({cos(\x)*1.5},{0.5+sin(\x)*0.5});
	\draw[thick, blue] plot[domain=-1.301:4.444, samples=150] ({\inverseellipseX}, {\inverseellipseY});
\end{tikzpicture}
}
	\only<5>{\begin{figure}
\begin{tikzpicture}[scale=.6]
	\drawgrid{-2.5}{-2.5}{2.5}{2.5}
	\draw[thick] (0, 0) circle [radius=1];
	\draw[thick, red] plot[domain=0:360, samples=40] ({cos(\x)*1.5},{0.5+sin(\x)*0.5});
	\draw[thick, blue] plot[domain=-1.301:4.444, samples=150] ({\inverseellipseX}, {\inverseellipseY});
\end{tikzpicture}
	\caption{ $E(x, y, z)$ }
	\label<5>{fig:ellipse5}
\end{figure}
}
	\only<6->{\begin{figure}
\begin{tikzpicture}[scale=.6]
	\drawgrid{-2.5}{-2.5}{2.5}{2.5}
	\draw[thick] (0, 0) circle [radius=1];
	\draw[thick, red] plot[domain=0:360, samples=40] ({cos(\x)*1.5},{0.5+sin(\x)*0.5});
	\draw[thick, blue] plot[domain=-1.301:4.444, samples=150] ({\inverseellipseX}, {\inverseellipseY});
\end{tikzpicture}
	\caption{ $E(x, y, z)$ }
	\label<6>{fig:ellipse6}
\end{figure}

Verweise auf {\tt label} mit {\tt ref}: \\
Siehe Abbildung \ref{fig:ellipse5},
	Tabelle \ref{tab:environments}
}
\end{overlayarea}
&
\begin{overlayarea}{.65\textwidth}{.8\textheight}
	\vspace{-1.5em}
	\only<1,2>{\lstinputlisting[style=myLaTeXstyleTiny]{tikz1.tex}}
	\only<2>{\lstinputlisting[style=myLaTeXstyleTiny]{tikzdef.tex}}
	\only<3>{\lstinputlisting[style=myLaTeXstyleTiny]{tikz12.tex}}
	\only<4>{\lstinputlisting[style=myLaTeXstyleTiny]{tikz123.tex}}
	\only<5>{\lstinputlisting[style=myLaTeXstyleTiny]{tikz1234.tex}}
	\only<6->{\lstinputlisting[style=myLaTeXstyleTiny]{tikz12345.tex}}
	\hypertarget<6>{tikz-final}{}
\end{overlayarea}
\end{tabular}
\end{frame}


\section{Weiteres}
\subsection{Was \LaTeX{} noch so kann}
% \demopageinput{advanced.tex}{.5\textwidth}{style=myLaTeXstyleTiny}
\demopageinput{pagenumbers.tex}{.5\textwidth}{style=myLaTeXstyleTiny}
\subsubsection{Altzwergisch}
\demopageinput{olddwarven.tex}{.5\textwidth}{style=myLaTeXstyleTiny}
\subsection{Referenzen und Templates}
\begin{frame}
\hspace{2em}
\begin{minipage}{.9\textwidth}
\begin{itemize}[<+->]
	\item[Slides] \url{https://tinyurl.com/ESELaTeX}
	\item[TUD] Templates der Uni
	\item[SE] \url{https://tex.stackexchange.com}
	\item[OL] \url{https://overleaf.com}
	\item[Tikz] \url{https://tikz.dev/}
	\item[Mathe] \url{https://de.wikipedia.org/wiki/Liste_mathematischer_Symbole}
	\item[Symbole] \url{https://latex-programming.fandom.com/wiki/List_of_LaTeX_symbols}
\end{itemize}
\end{minipage}
\end{frame}


%close the file that get the section page number definitions
\immediate\closeout\sectionpagenumbers

\end{document}
